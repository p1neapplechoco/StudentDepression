\documentclass[12pt,a4paper]{article}
\usepackage[utf8]{vietnam}
\usepackage{amsmath}
\usepackage{graphicx}
\usepackage{hyperref}
\usepackage{booktabs}
\usepackage{longtable}
\usepackage{array}
\usepackage{geometry}
\usepackage{xcolor}
\usepackage{colortbl}

\geometry{margin=2.5cm}

\title{\textbf{TEAM PLAN AND WORK DISTRIBUTION}\\
\large Student Depression Analysis Project}
\author{P4DS Course - University Data Science Program}
\date{December 8-22, 2025}

\begin{document}

\maketitle

\section{Team Member Information}

\begin{table}[h]
    \centering
    \begin{tabular}{|l|l|l|}
        \hline
        \textbf{Student ID} & \textbf{Full Name} & \textbf{Primary Responsibility}         \\
        \hline
        23122006            & Lưu Thượng Hồng    & Machine Learning \& Academic Analysis   \\
        \hline
        23122020            & Nguyễn Thiên Ấn    & Data Preprocessing \& Pressure Analysis \\
        \hline
        23122034            & Lê Nguyên Khang    & Data Exploration \& Lifestyle Analysis  \\
        \hline
    \end{tabular}
    \caption{Team Member Roles}
\end{table}

\section{Work Breakdown by Member}

\subsection{Lê Nguyên Khang (23122034) - 33.3\%}

\textbf{Primary Tasks:}
\begin{itemize}
    \item Initial data exploration and quality assessment
    \item Correlation analysis and statistical summaries
    \item Research Questions RQ1-RQ3 (Lifestyle Factors)
    \item Notebook: \texttt{01\_exploration.ipynb}
    \item Notebook: \texttt{02\_lifestyle.ipynb}
\end{itemize}

\textbf{Detailed Contributions:}
\begin{enumerate}
    \item \textbf{Data Exploration (Week 1: Dec 8-10)}
          \begin{itemize}
              \item Load and inspect dataset (27,901 records)
              \item Generate descriptive statistics for all features
              \item Create distribution plots for numerical variables
              \item Analyze categorical variable frequencies
              \item Check for missing values and duplicates
              \item Document initial findings in exploration notebook
          \end{itemize}

    \item \textbf{Correlation Analysis (Week 1: Dec 11-12)}
          \begin{itemize}
              \item Compute Pearson correlation matrix
              \item Create correlation heatmap visualization
              \item Identify relationships with Depression target
              \item Document surprising findings (CGPA correlation)
          \end{itemize}

    \item \textbf{RQ1: Sleep Paradox Analysis (Week 1: Dec 13-14)}
          \begin{itemize}
              \item Analyze depression rates across sleep duration groups
              \item Perform Chi-square test ($\chi^2$ = 9.07, p = 0.0026)
              \item Investigate reverse causation hypothesis
              \item Create visualizations comparing sleep groups
              \item Document findings: 7-8h (60.67\%) vs 5-6h (58.72\%)
          \end{itemize}

    \item \textbf{RQ2: Diet Compensation Effect (Week 2: Dec 15-17)}
          \begin{itemize}
              \item Conduct Kruskal-Wallis H-tests for each sleep group
              \item Analyze interaction between diet and sleep
              \item Calculate reduction percentages:
                    \begin{itemize}
                        \item <5h: 75.96\% $\rightarrow$ 50.38\% (25.58pp reduction)
                        \item 5-6h: 66.67\% $\rightarrow$ 48.28\% (18.39pp reduction)
                        \item >8h: 53.77\% $\rightarrow$ 36.64\% (17.13pp reduction)
                    \end{itemize}
              \item Create heatmap of Sleep $\times$ Diet interaction
          \end{itemize}

    \item \textbf{RQ3: Optimal Sweet Spot (Week 2: Dec 18-19)}
          \begin{itemize}
              \item Perform Mann-Whitney U test comparing best vs worst
              \item Calculate ROI for intervention pathways
              \item Document sweet spot: >8h + Healthy (36.64\%)
              \item Create intervention pathway visualizations
          \end{itemize}

    \item \textbf{Documentation \& Visualization (Week 2: Dec 20-21)}
          \begin{itemize}
              \item Polish all visualizations in lifestyle notebook
              \item Write comprehensive markdown explanations
              \item Ensure reproducibility of all analyses
              \item Contribute to final presentation slides
          \end{itemize}
\end{enumerate}

\textbf{Deliverables:}
\begin{itemize}
    \item \texttt{notebooks/01\_exploration.ipynb} - Complete EDA
    \item \texttt{notebooks/02\_lifestyle.ipynb} - RQ1-RQ3 analysis
    \item Statistical test results for lifestyle factors
    \item Visualizations: histograms, heatmaps, bar plots
\end{itemize}

\subsection{Nguyễn Thiên Ấn (23122020) - 33.3\%}

\textbf{Primary Tasks:}
\begin{itemize}
    \item Data cleaning and preprocessing pipeline
    \item Research Questions RQ4-RQ6 (Pressure Analysis)
    \item Notebook: \texttt{preprocessing.ipynb}
    \item Notebook: \texttt{03\_pressure.ipynb}
    \item Source: \texttt{src/preprocessing.py}
\end{itemize}

\textbf{Detailed Contributions:}
\begin{enumerate}
    \item \textbf{Data Preprocessing (Week 1: Dec 8-10)}
          \begin{itemize}
              \item Clean Sleep Duration field (remove single quotes)
              \item Drop ID column
              \item Implement ordinal encoding:
                    \begin{itemize}
                        \item Sleep Duration: 1-4 scale
                        \item Dietary Habits: 1-3 scale (Unhealthy to Healthy)
                    \end{itemize}
              \item Implement binary encoding:
                    \begin{itemize}
                        \item Suicidal Thoughts: Yes=1, No=0
                        \item Family History: Yes=1, No=0
                        \item Gender: Male=0, Female=1
                    \end{itemize}
              \item Create \texttt{preprocessing.py} module (252 lines)
              \item Write unit tests for encoding functions
          \end{itemize}

    \item \textbf{Preprocessing Pipeline Integration (Week 1: Dec 11-12)}
          \begin{itemize}
              \item Develop \texttt{preprocess\_pipeline()} entry point
              \item Create helper functions for data quality checks
              \item Implement \texttt{get\_missing\_values\_summary()}
              \item Implement \texttt{get\_duplicates\_count()}
              \item Document preprocessing notebook with examples
          \end{itemize}

    \item \textbf{RQ4: Cumulative Pressure Analysis (Week 1: Dec 13-15)}
          \begin{itemize}
              \item Build logistic regression model for suicidal thoughts
              \item Analyze three pressure types:
                    \begin{itemize}
                        \item Academic Pressure: $\beta$=0.377, OR=1.46, p<0.001
                        \item Financial Stress: $\beta$=0.270, OR=1.31, p<0.001
                        \item Work Pressure: $\beta$=0.189, OR=1.21, p=0.499 (NS)
                    \end{itemize}
              \item Test for exponential vs additive effects
              \item Create total pressure score variable
              \item Visualize pressure effect curves
          \end{itemize}

    \item \textbf{RQ5: Satisfaction as Protective Factor (Week 2: Dec 16-18)}
          \begin{itemize}
              \item Stratify data by Family History status
              \item Analyze Study Satisfaction levels (1-5)
              \item Create point plots with 95\% confidence intervals
              \item Document findings:
                    \begin{itemize}
                        \item Low satisfaction: 13pp gap between groups
                        \item High satisfaction: 2pp gap (not significant)
                    \end{itemize}
              \item Interpret protective factor mechanism
          \end{itemize}

    \item \textbf{RQ6: Danger Threshold Identification (Week 2: Dec 19-20)}
          \begin{itemize}
              \item Build predictive logistic model
              \item Calculate probability curves
              \item Identify 50\% threshold: Academic Pressure = 1.57
              \item Develop 3-tier intervention system:
                    \begin{itemize}
                        \item Green zone: 1.0-1.5 (<50\% risk)
                        \item Yellow zone: 1.6-3.0 (50-65\% risk)
                        \item Red zone: 3.1-5.0 (>65\% risk)
                    \end{itemize}
              \item Create threshold visualization
          \end{itemize}

    \item \textbf{Code Integration \& Testing (Week 2: Dec 21-22)}
          \begin{itemize}
              \item Integrate preprocessing into main pipeline
              \item Test end-to-end data flow
              \item Ensure compatibility with feature engineering
              \item Code review and refactoring
              \item Write inline documentation
          \end{itemize}
\end{enumerate}

\textbf{Deliverables:}
\begin{itemize}
    \item \texttt{src/preprocessing.py} - Complete preprocessing module
    \item \texttt{notebooks/preprocessing.ipynb} - Preprocessing demonstrations
    \item \texttt{notebooks/03\_pressure.ipynb} - RQ4-RQ6 analysis
    \item Logistic regression models with coefficients
    \item Threshold identification system
\end{itemize}

\subsection{Lưu Thượng Hồng (23122006) - 33.4\%}

\textbf{Primary Tasks:}
\begin{itemize}
    \item Machine learning model development
    \item Research Questions RQ7-RQ10 (Academic \& Work Analysis)
    \item Notebook: \texttt{04\_work.ipynb}
    \item Notebook: \texttt{modeling.ipynb}
    \item Source: \texttt{src/features.py}, \texttt{src/models.py}, \texttt{src/run\_pipeline.py}
\end{itemize}

\textbf{Detailed Contributions:}
\begin{enumerate}
    \item \textbf{Feature Engineering Development (Week 1: Dec 8-10)}
          \begin{itemize}
              \item Create age grouping function (6 brackets)
              \item Create CGPA grouping function (6 categories)
              \item Create satisfaction level groups (3 levels)
              \item Create pressure level groups (3 levels)
              \item Create work/study hours groups (4 brackets)
              \item Implement interaction term: AP $\times$ SS
              \item Develop achievement-satisfaction composite (5 categories)
              \item Create high achiever flag (CGPA $\geq$ 8.0)
              \item Write \texttt{features.py} module (181 lines)
          \end{itemize}

    \item \textbf{RQ7: Academic Factors Impact (Week 1: Dec 11-12)}
          \begin{itemize}
              \item Build logistic regression with CGPA and Hours
              \item Calculate correlations with Depression:
                    \begin{itemize}
                        \item Work/Study Hours: r=0.21, OR=1.38, p<0.001
                        \item CGPA: r=0.02, p>0.05 (not significant)
                    \end{itemize}
              \item Document "grades don't predict depression" finding
              \item Create scatter plots with regression lines
          \end{itemize}

    \item \textbf{RQ8: High-Risk Degree Programs (Week 1: Dec 13-14)}
          \begin{itemize}
              \item Calculate depression rates by degree (28 types)
              \item Identify highest risk: Class 12 (70.8\%, n=6,080)
              \item Perform statistical comparisons vs average (58.5\%)
              \item Investigate reasons for Class 12 risk:
                    \begin{itemize}
                        \item High-stakes university entrance exams
                        \item Intense peer competition
                        \item Parental expectation pressure
                        \item Developmental vulnerability (age 17-18)
                    \end{itemize}
              \item Create bar plot of depression by degree
          \end{itemize}

    \item \textbf{RQ9: Safe Threshold Analysis (Week 1: Dec 15-16)}
          \begin{itemize}
              \item Build predictive model for work/study hours
              \item Calculate probability curves
              \item Identify 50\% threshold: 4 hours/day
              \item Document depression rates by hour brackets:
                    \begin{itemize}
                        \item 0-3h: 50\%
                        \item 4-6h: 55\%
                        \item 7-9h: 62\%
                        \item 10-12h: 73\%
                    \end{itemize}
              \item Recommend safe zone: <4 hours/day
          \end{itemize}

    \item \textbf{RQ10: Achievement-Satisfaction Paradox (Week 2: Dec 17-18)}
          \begin{itemize}
              \item Create stratified analysis by CGPA and Satisfaction
              \item Calculate depression rates for 4 groups:
                    \begin{itemize}
                        \item Burnout (High CGPA + Low Sat): 67.4\%
                        \item At-Risk (Low CGPA + Low Sat): 67.6\%
                        \item Resilient (Low CGPA + High Sat): 52.9\%
                        \item Thriving (High CGPA + High Sat): 50.8\%
                    \end{itemize}
              \item Document 14.5pp advantage of satisfaction
              \item Create heatmap visualization
              \item Analyze "satisfaction trumps achievement" insight
          \end{itemize}

    \item \textbf{Machine Learning Pipeline (Week 2: Dec 19-20)}
          \begin{itemize}
              \item Implement 4 models:
                    \begin{itemize}
                        \item Logistic Regression (with scaling)
                        \item Random Forest (100 trees)
                        \item Gradient Boosting (100 estimators)
                        \item Decision Tree (max\_depth=5)
                    \end{itemize}
              \item Develop stratified 80-20 split
              \item Implement StandardScaler for LR
              \item Create evaluation framework (Accuracy, AUC, F1, Precision, Recall)
              \item Extract feature importance from tree models
              \item Write \texttt{models.py} module (251 lines)
          \end{itemize}

    \item \textbf{Model Evaluation \& Comparison (Week 2: Dec 21)}
          \begin{itemize}
              \item Train all 4 models
              \item Compare performance metrics
              \item Identify best model: Gradient Boosting (AUC=0.9245)
              \item Analyze feature importance rankings:
                    \begin{enumerate}
                        \item Suicidal Thoughts: 0.352
                        \item Academic Pressure: 0.198
                        \item Financial Stress: 0.156
                        \item Work/Study Hours: 0.089
                        \item Sleep Duration: 0.067
                    \end{enumerate}
              \item Validate RQ findings through feature importance
              \item Create model comparison visualizations
          \end{itemize}

    \item \textbf{Pipeline Integration \& Automation (Week 2: Dec 22)}
          \begin{itemize}
              \item Develop \texttt{run\_pipeline.py} (47 lines)
              \item Configure parameters: SMOTE, CV, test size
              \item Implement automated result saving
              \item Generate 3 output files:
                    \begin{itemize}
                        \item \texttt{model\_comparison.csv}
                        \item \texttt{feature\_importance.csv}
                        \item \texttt{processed\_data.csv}
                    \end{itemize}
              \item Test end-to-end pipeline execution
              \item Document usage in README
          \end{itemize}
\end{enumerate}

\textbf{Deliverables:}
\begin{itemize}
    \item \texttt{src/features.py} - Feature engineering module
    \item \texttt{src/models.py} - ML modeling module
    \item \texttt{src/run\_pipeline.py} - Main execution script
    \item \texttt{notebooks/04\_work.ipynb} - RQ7-RQ10 analysis
    \item \texttt{notebooks/modeling.ipynb} - Model development
    \item \texttt{results/} - All output CSV files
    \item Performance comparison tables
\end{itemize}

\section{Collaboration Process}

\subsection{Communication Channels}
\begin{itemize}
    \item \textbf{Version Control:} Git/GitHub for code collaboration
    \item \textbf{Meetings:} Twice weekly (Mon/Thu) via Zoom/in-person
    \item \textbf{Messaging:} Daily updates via group chat
    \item \textbf{Documentation:} Shared Google Drive for reports
\end{itemize}

\subsection{Workflow Management}
\begin{enumerate}
    \item \textbf{Week 1 (Dec 8-14): Individual Development}
          \begin{itemize}
              \item Each member focuses on primary responsibilities
              \item Daily stand-ups: progress updates, blockers, next steps
              \item Code commits to individual feature branches
              \item Peer review of completed notebooks
          \end{itemize}

    \item \textbf{Week 2 (Dec 15-21): Integration \& Refinement}
          \begin{itemize}
              \item Merge feature branches to main
              \item Resolve integration issues collectively
              \item Cross-review each other's work
              \item Ensure consistent coding style and documentation
              \item Test complete pipeline end-to-end
          \end{itemize}

    \item \textbf{Final Day (Dec 22): Finalization}
          \begin{itemize}
              \item Final code review and cleanup
              \item Complete README documentation
              \item Prepare presentation slides
              \item Practice presentation delivery
              \item Submit final deliverables
          \end{itemize}
\end{enumerate}

\subsection{Quality Assurance}
\begin{itemize}
    \item \textbf{Code Review:} All code reviewed by at least one other member
    \item \textbf{Testing:} Each module tested independently before integration
    \item \textbf{Documentation:} Inline comments and markdown explanations required
    \item \textbf{Reproducibility:} All analyses must run from scratch successfully
    \item \textbf{Style Guide:} Follow PEP 8 for Python code
\end{itemize}

\subsection{Shared Responsibilities}
While each member has primary tasks, the following are shared:
\begin{itemize}
    \item README.md completion (all contribute sections)
    \item Final presentation creation (collaborative slides)
    \item Data quality checks (all verify dataset integrity)
    \item Result interpretation (discuss findings together)
    \item Report writing (divide sections, peer edit)
\end{itemize}

\section{Project Planning and Timeline}

\subsection{Week 1: December 8-14, 2025}

\begin{longtable}{|p{2cm}|p{4cm}|p{4cm}|p{4cm}|}
    \hline
    \textbf{Date}                                        & \textbf{Khang (23122034)} & \textbf{An (23122020)} & \textbf{Hong (23122006)} \\
    \hline
    \endfirsthead
    \hline
    \textbf{Date}                                        & \textbf{Khang (23122034)} & \textbf{An (23122020)} & \textbf{Hong (23122006)} \\
    \hline
    \endhead

    \textbf{Dec 8 (Sun)}                                 &
    Load dataset, basic inspection                       &
    Clean Sleep Duration, drop ID column                 &
    Create age/CGPA grouping functions                                                                                                   \\
    \hline

    \textbf{Dec 9 (Mon)}                                 &
    Generate descriptive statistics, distribution plots  &
    Implement ordinal encoding (Sleep, Diet)             &
    Create satisfaction/pressure groups                                                                                                  \\
    \hline

    \textbf{Dec 10 (Tue)}                                &
    Analyze categorical variables, check quality         &
    Implement binary encoding (Gender, Suicidal, Family) &
    Implement interaction terms                                                                                                          \\
    \hline

    \textbf{Dec 11 (Wed)}                                &
    Compute correlation matrix                           &
    Develop preprocess\_pipeline() function              &
    Start RQ7: CGPA vs Hours analysis                                                                                                    \\
    \hline

    \textbf{Dec 12 (Thu)}                                &
    Create correlation heatmap, identify surprises       &
    Create data quality helper functions                 &
    Complete RQ7: logistic regression                                                                                                    \\
    \hline

    \textbf{Dec 13 (Fri)}                                &
    Start RQ1: Sleep paradox analysis                    &
    Start RQ4: Build pressure model                      &
    Start RQ8: Degree analysis                                                                                                           \\
    \hline

    \textbf{Dec 14 (Sat)}                                &
    Complete RQ1: Chi-square test, visualizations        &
    Continue RQ4: Analyze coefficients                   &
    Complete RQ8: Class 12 findings                                                                                                      \\
    \hline

    \multicolumn{4}{|c|}{\cellcolor{yellow}\textbf{Milestone 1: EDA \& Preprocessing Complete}}                                          \\
    \hline
\end{longtable}

\subsection{Week 2: December 15-22, 2025}

\begin{longtable}{|p{2cm}|p{4cm}|p{4cm}|p{4cm}|}
    \hline
    \textbf{Date}                            & \textbf{Khang (23122034)} & \textbf{An (23122020)} & \textbf{Hong (23122006)} \\
    \hline
    \endfirsthead
    \hline
    \textbf{Date}                            & \textbf{Khang (23122034)} & \textbf{An (23122020)} & \textbf{Hong (23122006)} \\
    \hline
    \endhead

    \textbf{Dec 15 (Sun)}                    &
    Start RQ2: Kruskal-Wallis tests          &
    Complete RQ4: Total pressure analysis    &
    Start RQ9: Hours threshold                                                                                               \\
    \hline

    \textbf{Dec 16 (Mon)}                    &
    Continue RQ2: Diet $\times$ Sleep interaction   &
    Start RQ5: Satisfaction stratification   &
    Complete RQ9: Probability curves                                                                                         \\
    \hline

    \textbf{Dec 17 (Tue)}                    &
    Complete RQ2: Reduction calculations     &
    Continue RQ5: Point plots with CI        &
    Start RQ10: Achievement-Sat groups                                                                                       \\
    \hline

    \textbf{Dec 18 (Wed)}                    &
    Start RQ3: ROI calculations              &
    Complete RQ5: Protective factor analysis &
    Complete RQ10: Paradox documentation                                                                                     \\
    \hline

    \textbf{Dec 19 (Thu)}                    &
    Complete RQ3: Sweet spot identification  &
    Start RQ6: Threshold model               &
    Start ML: Implement 4 models                                                                                             \\
    \hline

    \textbf{Dec 20 (Fri)}                    &
    Polish visualizations, markdown          &
    Complete RQ6: 3-tier system              &
    Train models, extract importance                                                                                         \\
    \hline

    \textbf{Dec 21 (Sat)}                    &
    Contribute to README, slides             &
    Integrate preprocessing, code review     &
    Complete modeling notebook, comparison                                                                                   \\
    \hline

    \textbf{Dec 22 (Sun)}                    &
    Final presentation prep                  &
    Final testing                            &
    Pipeline automation, finalize                                                                                            \\
    \hline

    \multicolumn{4}{|c|}{\cellcolor{green}\textbf{Milestone 2: Complete Project Delivery}}                                   \\
    \hline
\end{longtable}

\subsection{Key Milestones}

\begin{table}[h]
    \centering
    \begin{tabular}{|l|p{8cm}|l|}
        \hline
        \textbf{Milestone}                                   & \textbf{Deliverables} & \textbf{Date} \\
        \hline
        M1: EDA Complete                                     &
        \texttt{01\_exploration.ipynb}, correlation analysis &
        Dec 14                                                                                       \\
        \hline
        M2: Preprocessing Done                               &
        \texttt{preprocessing.py}, cleaned dataset           &
        Dec 14                                                                                       \\
        \hline
        M3: RQ1-RQ3 Complete                                 &
        \texttt{02\_lifestyle.ipynb}, statistical tests      &
        Dec 19                                                                                       \\
        \hline
        M4: RQ4-RQ6 Complete                                 &
        \texttt{03\_pressure.ipynb}, logistic models         &
        Dec 20                                                                                       \\
        \hline
        M5: RQ7-RQ10 Complete                                &
        \texttt{04\_work.ipynb}, feature engineering         &
        Dec 20                                                                                       \\
        \hline
        M6: ML Models Ready                                  &
        \texttt{modeling.ipynb}, \texttt{models.py}, results &
        Dec 21                                                                                       \\
        \hline
        M7: Pipeline Integrated                              &
        \texttt{run\_pipeline.py}, end-to-end testing        &
        Dec 22                                                                                       \\
        \hline
        M8: Documentation Complete                           &
        README.md, presentation slides                       &
        Dec 22                                                                                       \\
        \hline
        \textbf{Final Submission}                            &
        All code, notebooks, results, presentation           &
        \textbf{Dec 22}                                                                              \\
        \hline
    \end{tabular}
    \caption{Project Milestones and Deadlines}
\end{table}

\section{Risk Management}

\subsection{Potential Risks and Mitigation}

\begin{table}[h]
    \centering
    \small
    \begin{tabular}{|p{4cm}|p{3cm}|p{6cm}|}
        \hline
        \textbf{Risk}               & \textbf{Probability} & \textbf{Mitigation Strategy} \\
        \hline
        Integration conflicts       & Medium               &
        Daily commits, clear interfaces, integration testing on Dec 15                    \\
        \hline
        Time overrun on RQ analysis & Medium               &
        Buffer time in Week 2, prioritize core RQs (1,4,7,10)                             \\
        \hline
        Statistical test complexity & Low-Medium           &
        Use statsmodels/scipy libraries, consult documentation, peer review               \\
        \hline
        Model performance issues    & Low                  &
        Try multiple algorithms, tune hyperparameters, use cross-validation               \\
        \hline
        Code bugs in pipeline       & Medium               &
        Unit testing, code review, end-to-end testing Dec 22                              \\
        \hline
        Member unavailability       & Low                  &
        Cross-training on tasks, documentation of progress daily                          \\
        \hline
    \end{tabular}
    \caption{Risk Assessment and Mitigation}
\end{table}

\section{Expected Outcomes}

By December 22, 2025, the team will deliver:

\begin{enumerate}
    \item \textbf{Codebase}
          \begin{itemize}
              \item 4 Python modules: \texttt{preprocessing.py}, \texttt{features.py}, \texttt{models.py}, \texttt{run\_pipeline.py}
              \item 6 Jupyter notebooks with complete analyses
              \item Clean, documented, PEP 8 compliant code
              \item Git repository with commit history
          \end{itemize}

    \item \textbf{Analysis Results}
          \begin{itemize}
              \item Answers to all 10 research questions with statistical evidence
              \item Model comparison: 4 algorithms evaluated
              \item Feature importance rankings
              \item 3 CSV result files in \texttt{results/} directory
          \end{itemize}

    \item \textbf{Documentation}
          \begin{itemize}
              \item Comprehensive README.md (430 lines)
              \item Inline code comments and docstrings
              \item Markdown explanations in all notebooks
              \item This team plan document
          \end{itemize}

    \item \textbf{Presentation}
          \begin{itemize}
              \item 15-20 minute presentation slides
              \item Key findings and visualizations
              \item Live demo of pipeline execution
              \item Q\&A preparation
          \end{itemize}
\end{enumerate}

\section{Team Commitment}

All team members commit to:
\begin{itemize}
    \item Attend all scheduled meetings
    \item Complete assigned tasks on time
    \item Communicate proactively about challenges
    \item Support teammates when needed
    \item Maintain high code and analysis quality
    \item Contribute equally to final deliverables
\end{itemize}

\vspace{1cm}

\noindent\textbf{Signatures:}

\vspace{1cm}

\noindent\rule{4cm}{0.4pt} \hfill \rule{4cm}{0.4pt} \hfill \rule{4cm}{0.4pt}

\noindent Lê Nguyên Khang \hfill Nguyễn Thiên Ấn \hfill Lưu Thượng Hồng

\noindent 23122034 \hfill 23122020 \hfill 23122006

\vspace{0.5cm}

\noindent Date: December 8, 2025

\end{document}
